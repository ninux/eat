\section{Datenwiederherstellung}
Die Spannungsverläufe werden bei den Messungen als Bild (*.BMP) als auch als 
Messwerte (*.CSV) auf die CF Karte gespeichert. Bei der nachträglichen 
Auswertung wird festgestellt, dass einzelne Messreihen korrupt sind. 
Das Einlesen der Daten mittels \verb!dlmread! wird mit folgender Fehlermeldung 
abgebrochen: 
\lstinputlisting[breaklines=true, caption=Fehlermeldung]
{../data_corrupt/ALL0001/error.txt}
Wird die Datei in Excel importiert, fällt zunächst nur auf, dass die korrupte 
Datei weniger Linien enthält als eine normale Datei. Mit einem Editor wie Vim 
geöffnet, zeigt sich an der betroffenen Stelle folgendes Bild: 
\lstinputlisting[firstnumber=754,firstline=754,lastline=778,breaklines=true,
caption=Ausschnitt aus korrupten Daten wie mit VIM dargestellt]
{../data_corrupt/ALL0001/F0001CH1_corr.CSV}
\clearpage
Wird die entsprechende Stelle mit einem Hex Editor dargestellt, zeigt sich, 
dass die seltsamen Zeichen binäre Nullen sind: 
\lstinputlisting[firstnumber=1532,firstline=1532,lastline=1573,breaklines=true,
caption=Ausschnitt aus korrupten Daten dargestellt mit einem Hex Editor]
{../data_corrupt/ALL0001/F0001CH1.hex}
Wird die fehlerhafte Zeile gelöscht, fehlen an der entsprechenden Stelle 
Messwerte. Zudem sind die Vektoren der verschiedenen Kanäle nicht mehr gleich 
lang, was ein weiteres Verarbeiten erschwert. Daher werden die fehlenden 
Messwerte interpoliert. Damit können die Daten weiter verarbeitet werden. 
\lstinputlisting[firstnumber=41,firstline=41,lastline=41,breaklines=true,
caption=Interpolation der Messdaten]
{../matlab/gr631.m}
