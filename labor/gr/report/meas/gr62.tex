\section{Bestimmung der Glättungsinduktivität}
Der Wert der Glättungsinduktivität soll messtechnisch ermittelt werden. Dazu 
wird die Spannung bei einem Gleich- und Wechslestrom von jeweils 5 und 10 
Ampere gemessen. Aus den gemessenen Werten wird die Induktivität ermittelt. 
\begin{figure}[h!]
    \centering
    \begin{circuitikz}
        \draw (0,0) to[V] (0,3) to[short] (1,3) to [ammeter] (3,3) to[short] (5,3) to[L] (5,0) to[short] (0,0);
        \draw (3,3) to[voltmeter] (3,0);
    \end{circuitikz}
\end{figure}
\begin{table}[h!]
    \centering
    \begin{zebratabular}{lll}
    \rowcolor{gray}
    Messung / Berechneter Wert &
        Wert bei $I = 5\si{\ampere}$ &
        Wert bei $I = 10\si{\ampere}$ \\
    Spannung bei Wechselstrom 50\si{\hertz} $U_{AC}$ &
        87 \si{\volt} &
        172 \si{\volt} \\
    Spannung bei Gleichstrom $U_{DC}$ &
        0.86 \si{\volt} &
        1.72 \si{\volt} \\
    Impedanz $Z_L = \dfrac{U_{AC}}{I_{AC}}$ &
        17.2 \si{\ohm} &
        17.4 \si{\ohm} \\
    Widerstand $R_L = \dfrac{U_{DC}}{I_{DC}}$ &
        172 \si{\milli\ohm} &
        172 \si{\milli\ohm} \\
    Blindwiderstand $X_L = \sqrt{{Z_L}^2 - {R_L}^2}$ &
        17.3991 \si{\ohm} &
        17.1991 \si{\ohm} \\
    Induktivität $L = \dfrac{X_L}{2 \cdot \pi \cdot f}$ &
        55.38 \si{\milli\henry} &
        54.75 \si{\milli\henry} \\
    \end{zebratabular}
\end{table}
